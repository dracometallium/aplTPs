%%% LaTeX Template: Article/Thesis/etc. with colored headings and special fonts
%%%
%%% Source: http://www.howtotex.com/
% vim: set spell spelllang=es syntax=tex :

\documentclass[12pt]{article}

\usepackage{apuntes-estilo}
\usepackage{fancyhdr,lastpage}
\usepackage{graphicx}

\def\maketitle{

    \makeatletter
    {\color{bl} \centering \huge \sc \textbf{
    Trabajo práctico N° 6\\ Interacción entre diferentes Sistemas Operativos
    \large \vspace*{-8pt} \color{black} 
    \vspace*{8pt} }\par}
    \makeatother

    \makeatletter
    \input{tex/banner.tex}
}

% Custom headers and footers
\fancyhf{} % clear all header and footer fields
\fancypagestyle{plain}{\fancyhf{}}
\pagestyle{fancy}
\lhead{\footnotesize TP N° 6 - Interacción entre diferentes Sistemas Operativos }
\rhead{\footnotesize \thepage\ }	% "Page 1 of 2"

\def\ti#1#2{\texttt{#1} & #2 \\ }

\begin{document}

\thispagestyle{empty}
\maketitle
\setlength{\parindent}{0pt}

\section*{Introducción}

En una organización (aún cuando esta sea pequeña) podemos encontrarnos con
diferentes ambientes de trabajo, hardware y software heterogéneos, lo cual
supone varias dificultades.

En relación al software, los programas son diseñados para ciertos sistemas
operativos por lo cual no siempre funcionan en sistemas para los cuales no
fueron diseñados. Por ejemplo, los programas \emph{Windows} no ejecutan
correctamente en sistemas GNU/Linux ya que las llamadas al sistema y
bibliotecas no son las mismas. Análogamente ocurre con las aplicaciones
\emph{GNU/Linux}.

Aunque existen muchas alternativas libres al software que usamos a diario en
\emph{Windows}, y además muchas de las aplicaciones de uso común y más
populares cuentan con versiones para múltiples plataformas, más de una vez el
usuario se encuentra con que la mejor alternativa es la aplicación que solo
puede instalar en \emph{Windows}. Este suele ser el caso de las aplicaciones
gráficas, como por ejemplo \emph{Photoshop}.

\section*{Desafíos}

\begin{itemize}

    \item ¿Cómo ejecutamos aplicaciones \emph{Windows} en sistemas
        \emph{GNU/Linux}?

    \item ¿Cómo ejecutamos aplicaciones \emph{GNU/Linux} en sistemas
        \emph{Windows}?

\end{itemize}

\section*{Descripción}

Tomando como base una distribución \emph{Linux} y una distribución
\emph{Microsoft Windows}, se deberá elaborar un informe, según el esquema de la figura 1, sobre los siguientes puntos:

\begin{enumerate}

    \item Investigue, analice y compare,  las diferentes alternativas (libres
        y pagas)  para correr Máquinas Virtuales de un Sistema Operativo en
        otro diferente (\emph{Windows} en \emph{Linux y viceversa}).

    \item Explique el uso de las aplicaciones \emph{Wine} y \emph{Cygwin} y
        realice una comparación entre ambas.

    \item Analice las diferentes alternativas de escritorios remotos y de las
        herramientas/aplicaciones para logging remoto disponibles; teniendo en
        cuenta su utilización desde un Sistema Operativo al otro.

    \item Detalle en qué casos sugeriría cada una de las opciones anteriores.

    \item Realice una conclusión teniendo en cuenta las ventajas y desventajas
        de las opciones anteriores.

\end{enumerate}

%%%%%%%%%%%
La organización del trabajo recomendada es la siguiente:

\scalebox{0.39}{

\centering

\fbox{
\begin{minipage}[c][\paperheight]{\paperwidth}

\titlepage

\begin{center}
\ \\
\ \\
\vspace{-1cm}


\ \\

\vspace{0.5cm}
{\Large{\bf \sc Aplicaciones Libres}}\\

\ \\
{\Large { \sc Facultad de Informática}}\\

\ \\
{\Large{\bf \sc Universidad Nacional del Comahue}}\\


\vspace{-2.5cm}
\mbox{\hspace{-1cm}\includegraphics[width=2.5cm,height=2.5cm]{logos/uncoma.pdf}\hspace{13cm}
    \includegraphics[width=2.5cm,height=2.5cm]{logos/fai.pdf}}


\vspace{6cm}

{\Large {\bf\sc Trabajo Practico: Aplicaciones multimedia}}\\
\ \\
\ \\
\vspace{3cm}

{\Large Nombre, Apellido Autor1}\\
{\Large Nombre, Apellido Autor2}\\
{\Large Nombre, Apellido Autor3}\\

\vfill
{\Large fecha}\\

\end{center}

\end{minipage}
}
~
\fbox{
\begin{minipage}[c][\paperheight]{\paperwidth}

\section*{Indice:}

\begin{itemize}
    \item Título 1......... Nº de pag.
        \begin{itemize}
            \item Sección 1....... Nº de pag.
            \item Sección 2....... Nº de pag.
        \end{itemize}
    \item Título 2......... Nº de pag.
    \begin{itemize}
        \item Sección 1....... Nº de pag.
        \item Sección 2....... Nº de pag.
    \end{itemize}
\end{itemize}

\section*{Lista de gráficos:}

\begin{itemize}

    \item Título de figura o esquema 1..... Nº de pag.

    \item Título de figura o esquema 2..... Nº de pag.

\end{itemize}

\section*{Introducción}

Introducción al problema, importancia y objetivos.

\section*{Desarrollo:}
\begin{itemize}

    \item Toda información de importancia.

    \item Detallar explicar con vocabulario acorde. 

    \item Citar textos, poner opiniones de personas. Tiene que ser claro y
        preciso, también van a ir las imágenes y/o esquemas.

\end{itemize}

\section*{Conclusión:}

Retomar y analizar lo que se dijo previamente en el desarrollo y demostrar que
se cumplió con el objetivo y/o una conclusión final 

\section*{Bibliografía:}

Material bibliográfico detallado, si es un libro editorial, nombre del libro,
autor,etc. Si es pagina de Internet poner el link y fecha de consulta.

\end{minipage}
}
}

\end{document}
