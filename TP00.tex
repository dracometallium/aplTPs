%%% LaTeX Template: Article/Thesis/etc. with colored headings and special fonts
%%%
%%% Source: http://www.howtotex.com/
% vim: set spell spelllang=es syntax=tex :

\documentclass[12pt]{article}
\usepackage{apuntes-estilo}
\usepackage{fancyhdr,lastpage}
\usepackage{enumitem}

\newcommand{\code}[1]{\textbf{\tt #1}}

\def\maketitle{

\makeatletter
{\color{bl} \centering \huge \sc \textbf{ Trabajo práctico N° 0\\ \large
\vspace*{-8pt} \color{black}Repaso de conceptos previos\vspace*{8pt} }\par}
\makeatother

\makeatletter
\input{tex/banner.tex}
}

% Custom headers and footers
\fancyhf{} % clear all header and footer fields
\fancypagestyle{plain}{\fancyhf{}}
\pagestyle{fancy}
\lhead{\footnotesize TP N° 0 - Repaso de conceptos previos}
\rhead{\footnotesize \thepage\ }	% "Page 1 of 2"

\def\ti#1#2{\texttt{#1} & #2 \\ }

\begin{document}

\thispagestyle{empty}
\maketitle
\setlength{\parindent}{0pt}

\section*{Sistemas de archivos, tipos y formatos de archivos}

\begin{enumerate}

    \item \begin{enumerate}

        \item ¿Qué sistemas de archivos existen creados sobre memorias de
            almacenamiento masivo en la máquina actual? Lístelos incluyendo
            archivo de dispositivo, punto de montaje y sistema de archivos
            utilizado. \label{listar_sitemas_montados}

        \item   ¿Qué implica montar y desmontar un sistema de archivos? ¿Es
            posible desmontar el sistema de archivos cuyo punto de montaje es
            \code{/home}? ¿Cuál sería el riesgo de hacerlo? ¿Cuáles serial los
            pasos a seguir para intentarlo?

        \item ¿Para qué sirve el comando \code{tune2fs}? Utilizando dicho
            comando, guarde a continuación la salida de la opción \code{-l}
            para los sistemas de archivos listados en el punto
            \ref{listar_sitemas_montados}. Responda:

            \begin{enumerate}

                \item   ¿Cuántos \emph{inodos} disponibles existen? ¿Cuántos
                    de ellos están libres?

                \item   ¿Cuál es el tamaño de cada \emph{inodo}? ¿Dónde se
                    almacenan los mismos? ¿Qué conclusión puede sacar sobre
                    esto?

                \item   ¿Es el número de \emph{inodo} único en todo el
                    sistema? Ejecute por ejemplo \code{find / -inum 16}

            \end{enumerate}

        \item   ¿Es posible configurar el número de \emph{inodos} disponibles
            en un sistema de archivos al momento de crearlos? Observe el
            manual del comando \code{mkfs.ext4} por ejemplo.

        \item   ¿Para qué serviría? Piense en el tamaño promedio de los
            archivos almacenados en un sistema de archivos en particular.

        \end{enumerate}

    \item \begin{enumerate}

        \item   Utilice la opción \code{-type} del comando \code{find} para
            ubicar todos los archivos de cada tipo posible (utilice la página
            \code{man} de \code{find} para determinar los tipos posibles).
            Responda:

            \begin{enumerate}

                \item   ¿Cuántos archivos de cada tipo hay en el sistema?
                    Ordene por cantidad.

                \item   ¿Observa algún patrón de ubicación de cada tipo de
                    archivos?

                \item   ¿Cuántos tipos de archivos posibles existen en
                    GNU/Linux?

            \end{enumerate}

            \textbf{Ayuda1:} si lo prefiere, agregue la opción \code{-ls} para
            observar las propiedades de los archivos.

            \textbf{Ayuda2:} los mensajes de error puede ser obviados haciendo
            la redirección \code{2\textgreater/dev/null} al final del comando
            \code{find}.

        \item Repita la búsqueda por tipo de archivos pero esta vez aplique el
            comando file a cada archivo. Utilice la opción \code{-exec} de
            \code{find} para este propósito. Responda:

            \begin{enumerate}

                \item   ¿Para qué sirve el comando file? Describa brevemente
                    cómo funciona.

                \item   ¿Sobre qué tipo de archivos el resultado de file es
                    totalmente heterogéneo? ¿Tiene sentido?

            \end{enumerate}

    \end{enumerate}

    \item   Dentro del directorio home, crear un archivo de texto:

        \begin{enumerate}

            \item Observar las propiedades del archivo utilizando \code{stat}:
                ¿Cuál es el número de \emph{inodo}? ¿Cuántos enlaces existen a
                dicho archivo?

            \item Agregue dos nuevos enlaces fuertes/duros para el archivo
                recién creado. Vuelva a observar las propiedades de cada uno
                de ellos con \code{stat}.

            \item ¿Cuál es el número de \emph{inodo} para cada uno? ¿Que
                sucede con los tiempos de acceso? ¿Tamaño?

            \item Modifique el contenido de alguno de los archivos, observe
                qué sucede con el contenido de los otros enlaces ¿Qué
                conclusión saca?

            \item Elimine un enlace, verifique el decremento de la cuenta de
                enlaces con \code{stat} ¿Cuándo se libera efectivamente el
                espacio utilizado por el archivo?

            \item ¿Existe alguna forma de reconocer cuál fue el primer enlace
                creado?

        \end{enumerate}

    \item   Sobre los permisos:

        \begin{enumerate}

            \item ¿Cómo se agrupan los permisos de archivos? ¿Cuáles son los
                posibles permisos?

            \item ¿Quién puede modificar la propiedad, esto es dueño y grupo,
                de un archivo?

            \item ¿Existe desde la interfaz gráfica un modo en que los
                usuarios normales puedan observar/modificar los permisos y
                propiedad de archivos?

            \item ¿Que debería conocer un usuario normal acerca de dichos
                permisos?

        \end{enumerate}

    \item   ¿Qué sucede con las extensiones de archivos (\code{.pdf},
        \code{.txt}, etc.) en los sistemas \emph{GNU/Linux}? ¿El sistema hace
        alguna interpretación de las mismas? ¿El comando file utiliza las
        extensiones para determinar el tipo de archivos? ¿Es importante la
        extensión para las aplicaciones? Piense por ejemplo en las
        aplicaciones gráficas ¿Qué sucede si a un archivo, digamos
        \code{.pdf}, le borramos la extensión como parte del nombre?

\end{enumerate}

\end{document}
