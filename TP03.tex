%%% LaTeX Template: Article/Thesis/etc. with colored headings and special fonts
%%%
%%% Source: http://www.howtotex.com/
% vim: set spell spelllang=es syntax=tex :

\documentclass[12pt]{article}

\usepackage{apuntes-estilo}
\usepackage{fancyhdr,lastpage}
\usepackage{graphicx}

\def\maketitle{

\makeatletter
{\color{bl} \centering \huge \sc \textbf{
Trabajo práctico N° 3\\
\large \vspace*{-8pt} \color{black} Ofimática
\vspace*{8pt} }\par}
\makeatother

\makeatletter
\input{tex/banner.tex}

}

% Custom headers and footers
\fancyhf{} % clear all header and footer fields
\fancypagestyle{plain}{\fancyhf{}}
\pagestyle{fancy}
\lhead{\footnotesize TP N 3 - Ofimática}
\rhead{\footnotesize \thepage\ }	% "Page 1 of 2"

\def\ti#1#2{\texttt{#1} & #2 \\ }

\begin{document}

\thispagestyle{empty}
\maketitle
\setlength{\parindent}{0pt}

\section*{Ofimática}

Se deberá realizar un instructivo para orientar a los usuarios en el uso
correcto de las aplicaciones libres. Esta ayuda deberá documentarse en un
informe.

Cada grupo trabajará sobre un conjunto de situaciones que se puede presentar
ya sea en el uso de un procesador de texto, una planilla de cálculo o al
realizar una presentación. Deberá determinar para cada situación qué
característica de la aplicación necesita conocer el usuario y una explicación
con respecto a cómo debe usarla.

La explicación que se elabore debe ser en relación al uso de
\emph{LibreOffice} u \emph{OpenOffice}. En todos los casos, podrá utilizar
cualquier versión disponible, especificando la utilizada. Además deberá
incluir qué consideraciones de compatibilidad debe tener el usuario si la
información luego debe manipularse con \emph{MS Office}.

Para confeccionar la ayuda, se puede complementar la información textual con
los recursos que considere necesarios, como ejemplos, imágenes, etc.

\subsection*{Grupo 1: Procesador de Texto (LibreOffice Writer)}

\begin{description}

    \item{Usuario 1}: necesita que un documento sea revisado por otras 2
    personas, pero quiere ver los cambios que estos realicen. Necesita conocer
    cómo habilitar el control de cambios de modo que las otras dos personas
    sugieran cambios pero solo él los acepte o rechace, y cómo hacer para
    luego aceptarlos o rechazarlos para dejar el documento final.

    \item{Usuario 2}: necesita establecer una sangría de primera línea para
    cierto párrafo y un espaciado encima del párrafo de una $x$ cantidad de
    centímetros.

    \item{Usuario 3}: necesita separar el texto en dos columnas, estableciendo
    distinto ancho para las mismas y una separación de $xcm$ entre las mismas.
    Además necesita que el texto en las columnas tenga separación silábica.

    \item{Usuario 4}: necesita configurar el tamaño de la página, y el tamaño
    de los márgenes. Además necesita agregar un pie de página, configurándolo
    con una altura especifica, en el cual va a incluir la numeración
    automática de las páginas. También va a incorporar el encabezado, con una
    altura especifica en el cual quiere insertar una imagen desde un archivo.

    \item{Usuario 5}: necesita agregar una tabla a un documento. La misma es
    una nómina de los 15 empleados de la empresa, en la que deben constar los
    siguientes datos:

    \begin{itemize}
        \item Código
        \item Apellido
        \item Categoría
        \item Horas
        \item Precio Hora
    \end{itemize}

    El usuario necesita que la primera fila tenga una sola celda con un
    sombreado “gris claro”. El resto de las filas tendrá las cinco celdas para
    cada uno de los datos solicitados.

    \item{Usuario 6}: el usuario está trabajando con una tabla y necesita que
    el texto que está colocando en una celda tenga orientación vertical.
    Además, necesita agregar 5 filas a la tabla que editó antes y 1 columna
    más y consulta además cómo se elimina una fila o una columna. También
    necesita modificar el alto de una fila y el ancho de cada una de las
    columnas.

    \item{Usuario 7}: el usuario ha editado un texto y desea incorporar una
    imagen al mismo desde un archivo. La misma debe quedar contenida en el
    párrafo, en el lado derecho del mismo. Además, desea ampliar el tamaño de
    la misma, manteniendo la proporción y recortarle a la derecha unos
    centímetros. La imagen también deberá tener un borde color rojo de 2,5
    puntos.

    \item{Usuario 8}: esta editando un texto y necesita incorporar un esquema
    numerado con el siguiente formato:

    \begin{description}

        \item{Primer nivel}: con números del 1 en adelante, seguido por un
        guión.

        \item{Segundo nivel}: con letras mayúsculas, desde la A en adelante,
        seguido por un paréntesis de cierre.

        \item{Tercer nivel}: con numeración romana desde uno en adelante,
        seguido por un paréntesis de cierre.

        \item{Cuarto nivel}: con letras minúsculas, desde la a en adelante,
        seguido por un punto.

        \item{Quinto nivel}: con una viñeta. El símbolo es un asterisco.

    \end{description}

    \item{Usuario 9}: necesita incorporar un código de barras a su documento y
    no encuentra una fuente que le permite hacerlo.

    \item{Usuario 10}: al realizar informes habitualmente utiliza el mismo
    formato para las distintas partes que lo conforman, a saber, el título
    principal, títulos secundarios, párrafos simples, títulos en tablas, etc.
    Deberá explicarle como se define un estilo, como aplicarlo a la parte del
    documento que corresponda, como modificarlo y eliminarlo.

\end{description}

\subsection*{Grupo 2: Procesador de Texto (LibreOffice Writer)}

\begin{description}

    \item{Usuario 1}: necesita hacer una plantilla que contenga en el
    encabezado el membrete de la empresa, el cual incluye el nombre de la
    empresa, logotipo y dirección. Además la plantilla debe incluir una tabla.
    Además de crear la plantilla, el usuario necesita saber como utilizarla
    después para crear a partir de ella otros documentos.

    \item{Usuario 2}: necesita agregar en su documento un hipervinculo a una
    página .

    \item{Usuario 3}: en su equipo tiene instalado \emph{MS Office} y ahora
    dado que se planea la migración a alguna de las aplicaciones libres y ha
    comenzado a utilizarla consulta sobre los atajos, si son los mismos o son
    diferentes.

    \item{Usuario 4}: necesita crear un conjunto de cartas que tienen la misma
    información, lo único que se modifica es a quién va dirigido. La
    información de las empresas (Razón Social, Dirección, Responsable) a las
    que se necesita enviar la carta se encuentra en una planilla de cálculo.
    Además de la nota, necesita hacer el sobre correspondiente a cada carta a
    enviar. Para hacer esto, consulta sobre como crear e imprimir el sobre,
    como realizar el documento base de la carta, el manejo de la fuente de
    datos y como hacer para realizar automáticamente todas las cartas e
    imprimirlas.

    \item{Usuario 5}: necesita establecer distintos encabezados y pie de
    página en las páginas pares e impares.

    \item{Usuario 6}: necesita crear un documento con distintas secciones.
    Cada una de ellas tiene un encabezado de página diferente, en donde se
    hace referencia al título de la sección, mientras que el pie de página es
    igual para todas ya que solo contiene el número de página.

    \item{Usuario 7}: consulta como debe hacer para cambiar el tipo de letra,
    su tamaño, color, y subrayado.

    \item{Usuario 8}: necesita comparar dos versiones de un mismo documento y
    ver las diferencias entre los mismos.

    \item{Usuario 9}: esta editando un texto y necesita incorporar una lista
    con viñetas utilizando como símbolo un rombo. Además quiere conocer como
    quitar el formato de lista con viñetas.

    \item{Usuario 10}: era un usuario de \emph{MS Office} y ahora que esta
    utilizando el procesador de texto libre, no encuentra las fuentes que
    habitualmente usa, tales como \emph{Time New Roman} y \emph{Arial}.

\end{description}

\subsection*{Grupo 3: Planilla de Cálculo (LibreOffice Calc)}

\begin{description}

    \item{Usuario 1}: necesita ajustar automáticamente el ancho de una columna
    para que se adapte al texto más largo y el alto de una fila. Además
    consulta sobre como modificar manualmente el tamaño de las filas y las
    columnas.

    \item{Usuario 2}: necesita que el texto en una celda se adapte al tamaño
    de la misma de dos maneras diferentes. Por un lado reduciendo el tamaño de
    la fuente automáticamente y por otro lado haciendo que el texto se ajuste
    al ancho por medio de dividirse en varias líneas.

    \item{Usuario 3}: necesita crear una fórmula en la que se sumen dos
    valores que se encuentra en dos celdas adyacentes y que dicha fórmula se
    aplique en todas las filas siguientes. Por otro lado necesita agregar una
    fórmula en la que se multipliquen dos valores que se encuentran en
    distintas celdas. La fórmula deberá luego aplicarse a todas las filas
    siguientes, pero manteniendo fija la referencia a una de las celdas
    referenciadas en la fórmula.

    \item{Usuario 4}: necesita agregar una columna entre medio de dos columnas
    que poseen datos. También debe agregar una fila entre medio de otras con
    información.

    \item{Usuario 5}: necesita establecer un formato específico para una celda
    dependiendo del valor que tenga la misma, de manera que la misma tenga un
    sombreado de color y un tamaño y tipo de fuente determinada. También
    consulta sobre como eliminar el formato aplicado.

    \item{Usuario 6}: necesita contar cierta cantidad de celdas en un rango
    que cumplan una determinada condición. Además necesita sumar el valor de
    las celdas en un rango que cumplen una condición.

    \item{Usuario 7}: dada la información en una hoja, necesita realizar un
    gráfico del tipo columnas agrupadas con efecto 3D. No necesita mostrar
    leyendas y quiere agregar un título acorde. Además el gráfico debe estar
    en una hoja nueva.

    \item{Usuario 8}: necesita escribir en una columna una serie de números
    cuyo incremento es de 10 en 10. Por otro lado en otra columna necesita
    escribir una serie de fechas cuyo incremento es de uno en uno. Además,
    necesita crear otra serie en la que cada elemento sea el doble del
    anterior.

    \item{Usuario 9}: tiene una hoja de cálculo en las que hay ciertos datos
    críticos que no deben ser modificados, para lo cual necesita protegerla.

    \item{Usuario 10}: era un usuario de \emph{MS Office} y ahora que esta utilizando
    una aplicación libre de planillas de cálculo, no puede cambiar el tipo de
    referencia (absoluta, relativa o mixta) presionando la tecla F4.

\end{description}

\subsection*{Grupo 4: Planilla de Cálculo (LibreOffice Calc)}

\begin{description}

    \item{Usuario 1}: necesita combinar o unir 5 celdas en una fila.

    \item{Usuario 2}: tiene en una hoja ciertos datos para los cuales quiere
    impedir que otros vean las fórmulas que ha aplicado.

    \item{Usuario 3}: necesita agregar más hojas de cálculo y asignarles un
    nombre a las mismas. Además quiere cambiar de posición una de ellas.

    \item{Usuario 4}: tiene cierta información de la cual solo quiere
    visualizar las filas en las que se cumple una determinada condición en una
    de las columnas. Por otro lado, en otra planilla quiere visualizar solo
    las filas en las que el valor de una columna esta en un rango de valores
    específicos.

    \item{Usuario 5}: necesita ordenar los datos que tiene en una planilla de
    acuerdo al contenido de dos columnas.

    \item{Usuario 6}: necesita establecer que al imprimir, en cada hoja se va
    a repetir como título lo contenido en una fila.  Usuario 7: necesita
    restringir la entrada de datos en un conjunto de celdas a un intervalo
    determinado de fechas.

    \item{Usuario 8}: necesita realizar un análisis de los datos en una
    planilla. Para ello sin insertar fórmulas manualmente, quiere resumir los
    datos mostrando subtotales y totales de la información, usando sumas,
    promedios, mínimos y máximos.

    \item{Usuario 9}: necesita realizar un análisis de los datos en una
    planilla, organizando y resumiendo los datos según diferentes criterios en
    forma interactiva. Para esto debe utilizar Tablas Dinámicas pero no
    entiende como hacerlas y actualizarlas cuando la fuente de información se
    modifica.

    \item{Usuario 10}: era un usuario de \emph{MS Office} y ahora que esta utilizando
    una aplicación libre de planillas de cálculo, no encuentra las fuentes que
    habitualmente usa, tales como \emph{Time New Roman} y \emph{Arial}.

\end{description}

\subsection*{Grupo 5: Presentación (LibreOffice Impress)}

\begin{description}

    \item{Usuario 1}: necesita definir para todas las diapositivas que van a
    conformar su presentación, el fondo, el color, el tipo de fuentes y el pie
    de páginas.

    \item{Usuario 2}: necesita saber como agregar, eliminar y duplicar una
    diapositiva.

    \item{Usuario 3}: necesita agregar un efecto de transición entre
    diapositivas de la presentación. Además quiere agregar efectos a cada
    elemento de una diapositiva.

    \item{Usuario 4}: necesita agregar a su presentación varias imágenes desde
    un archivo, modificar el tamaño y la posición de las mismas. En uno de los
    casos también va a ser necesario que recorte la imagen.

    \item{Usuario 5}: necesita incorporar una lista con viñetas utilizando
    como símbolo un asterisco. Además quiere conocer como quitar el formato de
    lista con viñetas.

    \item{Usuario 6}: necesita reproducir un audio durante toda la
    presentación.

    \item{Usuario 7}: tiene una diapositiva en la que los distintos elementos
    se encuentran uno encima del otro y necesita ordenarlos. Algunos de ellos
    tienen que verse adelante de otros.

    \item{Usuario 8}: necesita agregar en una diapositiva un cuadro de texto
    al que deberá asignarle un color de relleno, cambiar el grosor del tipo de
    línea y establecer un tipo de fuente determinada. También consulta sobre
    como se agrega el texto al cuadro.

    \item{Usuario 9}: necesita agregar una línea cuyos dos extremos finalicen
    con una punta de flecha. Además quiere establecer un color para la flecha
    y modificar el grosor de la misma.

    \item{Usuario 10}: necesita saber para que sirven la distintas vistas de
    la presentación (Normal, Esquema, etc). Además quiere saber como debe
    hacer para que su presentación pueda abrirse en cualquier otro equipo.

\end{description}

%%%%%%%%%%%

La organización del trabajo recomendada es la siguiente:

\scalebox{0.39}{

\centering

\fbox{
\begin{minipage}[c][\paperheight]{\paperwidth}

\titlepage

\begin{center}
\ \\
\ \\
\vspace{-1cm}


\ \\

\vspace{0.5cm}
{\Large{\bf \sc Aplicaciones Libres}}\\

\ \\
{\Large { \sc Facultad de Informática}}\\

\ \\
{\Large{\bf \sc Universidad Nacional del Comahue}}\\


\vspace{-2.5cm}
\mbox{\hspace{-1cm}\includegraphics[width=2.5cm,height=2.5cm]{logos/uncoma.pdf}\hspace{13cm}
    \includegraphics[width=2.5cm,height=2.5cm]{logos/fai.pdf}}


\vspace{6cm}

{\Large {\bf\sc Trabajo Practico: Ofimática}}\\
\ \\
\ \\
\vspace{3cm}

{\Large Nombre, Apellido Autor1}\\
{\Large Nombre, Apellido Autor2}\\
{\Large Nombre, Apellido Autor3}\\

\vfill
{\Large fecha}\\

\end{center}

\end{minipage}
}
~
\fbox{
\begin{minipage}[c][\paperheight]{\paperwidth}

\section*{Indice:}

\begin{itemize}
	\item Título 1......... Nº de pag.
		\begin{itemize}
			\item Sección 1....... Nº de pag.
			\item Sección 2....... Nº de pag.
		\end{itemize}
	\item Título 2......... Nº de pag.
		\begin{itemize}
			\item Sección 1....... Nº de pag.
			\item Sección 2....... Nº de pag.
		\end{itemize}
\end{itemize}

\section*{Lista de gráficos:}

\begin{itemize}

	\item Título de figura o esquema 1..... Nº de pag.

	\item Título de figura o esquema 2..... Nº de pag.

\end{itemize}

\section*{Introducción}

Introducción al problema, importancia y objetivos.

\section*{Desarrollo:}
\begin{itemize}

	\item Toda información de importancia.

	\item Detallar explicar con vocabulario acorde. 

	\item Citar textos, poner opiniones de personas. Tiene que ser claro y
		preciso, también van a ir las imágenes y/o esquemas.

\end{itemize}

\section*{Conclusión:}

Retomar y analizar lo que se dijo previamente en el desarrollo y demostrar que
se cumplió con el objetivo y/o una conclusión final 

\section*{Bibliografía:}

Material bibliográfico detallado, si es un libro editorial, nombre del libro,
autor,etc. Si es pagina de Internet poner el link y fecha de consulta. 

\end{minipage}
}

}
 
\end{document}
