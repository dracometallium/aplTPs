%%% LaTeX Template: Article/Thesis/etc. with colored headings and special fonts
%%%
%%% Source: http://www.howtotex.com/
% vim: set spell spelllang=es syntax=tex :

\documentclass[12pt]{article}

\usepackage{apuntes-estilo}
\usepackage{fancyhdr,lastpage}
\usepackage{graphicx}
\usepackage{hyperref}
\usepackage{xurl}

\def\maketitle{

    \makeatletter
    {\color{bl} \centering \huge \sc \textbf{
    Trabajo práctico N° 4\\
    \large \vspace*{-8pt} \color{black} Aspectos legales de las producciones
    Multimedia
    \vspace*{8pt} }\par}
    \makeatother

    \makeatletter
    \input{tex/banner.tex}
}

% Custom headers and footers
\fancyhf{} % clear all header and footer fields
\fancypagestyle{plain}{\fancyhf{}}
\pagestyle{fancy}
\lhead{\footnotesize TP N° 4 - Aspectos legales de las producciones
Multimedia}
\rhead{\footnotesize \thepage\ }	% "Page 1 of 2"

\def\ti#1#2{\texttt{#1} & #2 \\ }

\begin{document}

\thispagestyle{empty}
\maketitle
\setlength{\parindent}{0pt}

\section*{Introducción}

Las composiciones multimedia, traen aparejado la necesidad de evaluar ciertos
aspectos legales que pueden afectar a la organización que la produce (o
individuo). Por ejemplo, una presentación como las que podemos realizar con
\emph{LibreOffice Impress}, podría traer aparejado un conflicto legal para la
organización si su contenido digital no esta correctamente verificado.

En grandes organizaciones, existen departamentos especializados en lo
referente a aspectos legales, que establecen las políticas a seguir cuando los
empleados pretenden crear contenidos multimedia en nombre de la misma. De ese
modo, los usuarios deberán respetar tal o cual licencia sobre su producción,
podrán o no utilizar el logo de la empresa, etc. Un empleado podría
preguntarse por ejemplo, si puede utilizar un video en su presentación, que
acaba de descargar de la web.

En organizaciones más pequeñas, muchas veces este departamento especializado
no existe, y las producciones multimedia de los usuarios se crean sin ningún
tipo de asesoramiento legal. En estos casos, muchas veces recae en el
administrador o gerente de sistemas, alertar a los usuarios sobre las
conductas que pueden generar un conflicto legal a la hora de crear
producciones multimedia. Porque después de todo, fueron los administradores
los que instalaron el software que terminó dando origen a la producción
(aunque lamentable, muchas veces esta es la asociación que padecen los
administradores).

Por un lado está la legalidad del software utilizado para la composición
multimedia. Este aspecto legal ha sido cubierto en otras materias, y atañe
tanto a administradores como a usuarios. En líneas generales, debemos respetar
cada licencia de cada software utilizado para la composición multimedia (desde
los editores, reproductores, software de autoría etc).

Pero más complejo aún es lo referente a los derechos de autor sobre la
composición multimedia en sí misma, y cada elemento individual que la compone:
texto, imagen, video, etc.

\section*{Descripción}

La intención de esta monografía es abrir el debate para alertarnos acerca de
la complejidad de este problema y vislumbrar ideas acerca de cómo prevenir
conflictos legales derivados de las producciones multimedia. Como punto de
partida, se recomienda analizar el documental producido por \emph{\textbf{TVE}
(Televisión Española)}, llamado \emph{Copiad Malditos}:
\url{http://copiadmalditos.blogspot.com.ar/p/videos-el-documental.html}

El trabajo deberá responder las siguientes preguntas:

\begin{itemize}

    \item ¿Qué es el derecho de autor?
      
    \item ¿Qué es el copyright?
      
    \item ¿Qué es el \emph{copyleft}?
      
    \item ¿Cuál es el organismo nacional equivalente a SGAE de España?
      
    \item ¿Qué es \emph{creative commons}?
      
    \item ¿Qué tipos de licencias \emph{creative commons} existen?

    \item ¿Qué implica cada tipo de licencia \emph{creative commons}?
      
    \item ¿Hay otras licencias de tipo \emph{copyleft} para producciones
        multimedia?
      
    \item ¿Qué cree que sucede con los derechos de auto dentro de las grandes
        corporaciones y empresas privadas?
      
    \item En el caso de la producción multimedia del documental de TVE,
        ¿Importa sólo el derecho de autor de quien hace la composición en su
        conjunto o se ven involucrados los derechos de los productores de cada
        elemento individual?
      
    \item Piense en sus propias producciones multimedia, por ejemplo todos los
        documentos creados para esta materia y subidos a la PedCO. ¿Tiene Ud.
        derecho sobre cada imagen, texto, video, etc; utilizado en su
        producción? ¿Pensó que podría producir un conflicto a la Universidad
        por utilizar elementos sobre los cuales no tiene derechos? Observe que
        la PedCO esta disponible al público en general a través de Internet,
        bajo un dominio controlado por la Universidad.

\end{itemize}

%%%%%%%%%%%
La organización del trabajo recomendada es la siguiente:

\scalebox{0.39}{

\centering

\fbox{
\begin{minipage}[c][\paperheight]{\paperwidth}

\titlepage

\begin{center}
\ \\
\ \\
\vspace{-1cm}


\ \\

\vspace{0.5cm}
{\Large{\bf \sc Aplicaciones Libres}}\\

\ \\
{\Large { \sc Facultad de Informática}}\\

\ \\
{\Large{\bf \sc Universidad Nacional del Comahue}}\\


\vspace{-2.5cm}
\mbox{\hspace{-1cm}\includegraphics[width=2.5cm,height=2.5cm]{logos/uncoma.pdf}\hspace{13cm}
    \includegraphics[width=2.5cm,height=2.5cm]{logos/fai.pdf}}


\vspace{6cm}

{\Large {\bf\sc Trabajo Practico: Aplicaciones multimedia}}\\
\ \\
\ \\
\vspace{3cm}

{\Large Nombre, Apellido Autor1}\\
{\Large Nombre, Apellido Autor2}\\
{\Large Nombre, Apellido Autor3}\\

\vfill
{\Large fecha}\\

\end{center}

\end{minipage}
}
~
\fbox{
\begin{minipage}[c][\paperheight]{\paperwidth}

\section*{Indice:}

\begin{itemize}
    \item Título 1......... Nº de pag.
        \begin{itemize}
            \item Sección 1....... Nº de pag.
            \item Sección 2....... Nº de pag.
        \end{itemize}
    \item Título 2......... Nº de pag.
    \begin{itemize}
        \item Sección 1....... Nº de pag.
        \item Sección 2....... Nº de pag.
    \end{itemize}
\end{itemize}

\section*{Lista de gráficos:}

\begin{itemize}

    \item Título de figura o esquema 1..... Nº de pag.

    \item Título de figura o esquema 2..... Nº de pag.

\end{itemize}

\section*{Introducción}

Introducción al problema, importancia y objetivos.

\section*{Desarrollo:}
\begin{itemize}

    \item Toda información de importancia.

    \item Detallar explicar con vocabulario acorde. 

    \item Citar textos, poner opiniones de personas. Tiene que ser claro y
        preciso, también van a ir las imágenes y/o esquemas.

\end{itemize}

\section*{Conclusión:}

Retomar y analizar lo que se dijo previamente en el desarrollo y demostrar que
se cumplió con el objetivo y/o una conclusión final 

\section*{Bibliografía:}

Material bibliográfico detallado, si es un libro editorial, nombre del libro,
autor,etc. Si es pagina de Internet poner el link y fecha de consulta.

\end{minipage}
}
}

\end{document}
