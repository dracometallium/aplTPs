%%% LaTeX Template: Article/Thesis/etc. with colored headings and special fonts
%%%
%%% Source: http://www.howtotex.com/
% vim: set spell spelllang=es syntax=tex :

\documentclass[12pt]{article}

\usepackage{apuntes-estilo}
\usepackage{fancyhdr,lastpage}
\usepackage{graphicx}

\def\maketitle{

    \makeatletter
    {\color{bl} \centering \huge \sc \textbf{
    Trabajo práctico N° 4\\
    \large \vspace*{-8pt} \color{black} Aplicaciones Multimedia
    \vspace*{8pt} }\par}
    \makeatother

    \makeatletter
    \input{tex/banner.tex}

}

% Custom headers and footers
\fancyhf{} % clear all header and footer fields
\fancypagestyle{plain}{\fancyhf{}}
\pagestyle{fancy}
\lhead{\footnotesize TP N° 4 - Aplicaciones Libres}
\rhead{\footnotesize \thepage\ }	% "Page 1 of 2"

\def\ti#1#2{\texttt{#1} & #2 \\ }

\begin{document}

\thispagestyle{empty}
\maketitle
\setlength{\parindent}{0pt}

\section*{Introducción}


El mundo de las aplicaciones multimedia es inmenso. Como administradores de
sistemas encontraremos cientos de requisitos diferentes sobre imagen, audio,
video y autoría multimedia.

Este trabajo tendrá varios objetivos con respecto a multimedia:

\begin{itemize}

\item Conocer terminología básica respecto a cada componente multimedia.

    \item Conocer la problemática básica de las aplicaciones. Cómo se agrupan,
    cuáles son las funcionalidades principales.

    \item Visualizar los recursos necesarios a nivel de hardware y software.

    \item Vincular aplicaciones con los formatos de archivos manipulados.

    \item Vincular aplicaciones de software libre con aplicaciones utilizadas
    como estándar de facto, posiblemente privativas.

\end{itemize}

\section{Descripción}

Es habitual que comunidades de personas con fines comunes aúnen esfuerzo para
resolver ciertos tipo de problemas, el mundo del software, y en particular el
mundo de los sistemas operativos GNU/Linux no es ajeno a este fenómeno. En
este caso, y con los objetivos mencionados anteriormente, analizaremos el
trabajo de comunidades de usuarios y desarrolladores de aplicaciones
multimedia, que han dado como resultado nuevas distribuciones GNU/Linux.

Nuestro objetivo no será promover el uso de las distribuciones en si, sino el
análisis de las mismas como medio de obtener conocimiento acerca de
aplicaciones multimedia instaladas en ellas y la organización de las mismas
(forma en que están agrupadas y presentadas a los usuarios). El administrador
podrá considerar en el futuro este tipo de soluciones cuando se adapte a los
requisitos de los usuarios.

A cada grupo se le asignara una distribución sobre la que deberán realizar un
informe cubriendo todos los puntos que se plantean más adelante. Se debe
respetar el mismo formato que en los trabajos anteriores. Soló uno de los
integrantes debe subir el informe completo, el resto debe subir únicamente la
caratula.

Las distribuciones a considerar serán:

\begin{itemize}

    \item Ubuntu Studio: http://ubuntustudio.org/

    \item DreamStudio: (formerly Dream Studio) http://dreamstudio.com/

    \item OpenArtist: http://openartisthq.org/

    \item Musix GNU+Linux: https://musixdistro.wordpress.com/

    \item AV Linux: http://www.bandshed.net/avlinux/

\end{itemize}

\subsubsection{Características básicas de la distribución}

La siguiente información deberá ser recolectada para la distribución asignada:

\begin{enumerate}

    \item País de origen.

    \item Logo que la representa.

    \item ¿Deriva de alguna otra distribución?

    \item ¿Se encuentra activo su desarrollo? ¿De cuándo es la fecha de su
    última versión estable?

    \item ¿Qué entorno gráfico utiliza?

    \item ¿Cuál es su página web oficial?

    \item ¿Para qué arquitectura de hardware se encuentra disponible?

    \item ¿Proveen software privativo como parte de la instalación?

    \item ¿Cómo están organizadas las aplicaciones? ¿Es sencillo encontrar
    aplicaciones multimedia para cada fin? Por ejemplo, si queremos un editor
    de sonido.

\end{enumerate}

\subsubsection*{Aplicaciones multimedia}

Recorra la distribución y elija aplicaciones multimedia que satisfagan las
siguientes características:

\begin{itemize}

    \item 4 aplicaciones relacionadas a imágenes estáticas bidimensionales.

    \item 4 aplicaciones relacionadas a sonido

    \item 4 aplicaciones relacionadas a video y animación.

    \item 2 aplicaciones relacionadas a autoría multimedia.

\end{itemize}

Las aplicaciones elegidas no deben ser obsoletas, es decir deben tener una
comunidad activa y desarrollo no superior a tres años, es decir la última
versión disponible por el desarrollador no debe superar los tres años.

Para cada aplicación deberá:

\begin{enumerate}

    \item Describir brevemente su funcionalidad.

    \item Características técnicas: Requisitos generales de sistema:
    almacenamiento masivo, RAM, CPU, hardware específico, etc. Versión.
    Licencia. Sitio del desarrollo (\emph{homepage/mainstream}).

    \item Formatos de datos manipulados:

    \begin{itemize}

        \item Indique qué formatos manipula la aplicación.

        \item Indique si los formatos son nativos de la aplicación.

        \item De ser posible indique si los formatos aplican compresión o no.

        \item En el caso de formatos de imágenes 2D estáticas, indique si son
            vectoriales o matriciales (mapa de bits).


    \end{itemize}

    \item Dar dos ejemplos de uso de la aplicación. ¿Para qué podría servir?

    \item ¿Existe alguna aplicación que sea un estándar de facto y cubra esta
    funcionalidad? Por ejemplo el estándar de facto para dibujo CAD suele ser
    \emph{Autocad (R) de Autodesk}, si bien existe \emph{FreeCad}, \emph{Qcad}
    y otros que tienen funcionalidades similares.

    \item En el caso de aplicaciones de autoría: ¿Permiten agregar
    interactividad a la composición? ¿Son lineales o no lineales? ¿Cómo podría
    un usuario hacer uso de la composición resultante?

\end{enumerate}

\subsection*{Preguntas generales}

\begin{itemize}

    \item ¿Cuál es el modelo de la placa de video? ¿Cómo obtiene esta
    información?

    \item ¿Cuál es el modelo de la placa de sonido? ¿Cómo obtiene esta
    información?

    \item ¿Cuál es el driver utilizado para la tarjeta de video?

    \item ¿Cuál es el driver utilizado para la tarjeta de sonido?

\end{itemize}

La organización del trabajo recomendada es la siguiente:

\scalebox{0.39}{

\centering

\fbox{
\begin{minipage}[c][\paperheight]{\paperwidth}

\titlepage

\begin{center}
\ \\
\ \\
\vspace{-1cm}


\ \\

\vspace{0.5cm}
{\Large{\bf \sc Aplicaciones Libres}}\\

\ \\
{\Large { \sc Facultad de Informática}}\\

\ \\
{\Large{\bf \sc Universidad Nacional del Comahue}}\\


\vspace{-2.5cm}
\mbox{\hspace{-1cm}\includegraphics[width=2.5cm,height=2.5cm]{logos/uncoma.pdf}\hspace{13cm}
    \includegraphics[width=2.5cm,height=2.5cm]{logos/fai.pdf}}


\vspace{6cm}

{\Large {\bf\sc Trabajo Practico: Aplicaciones multimedia}}\\
\ \\
\ \\
\vspace{3cm}

{\Large Nombre, Apellido Autor1}\\
{\Large Nombre, Apellido Autor2}\\
{\Large Nombre, Apellido Autor3}\\

\vfill
{\Large fecha}\\

\end{center}

\end{minipage}
}
~
\fbox{
\begin{minipage}[c][\paperheight]{\paperwidth}

\section*{Indice:}

\begin{itemize}
    \item Título 1......... Nº de pag.
        \begin{itemize}
            \item Sección 1....... Nº de pag.
            \item Sección 2....... Nº de pag.
        \end{itemize}
    \item Título 2......... Nº de pag.
    \begin{itemize}
        \item Sección 1....... Nº de pag.
        \item Sección 2....... Nº de pag.
    \end{itemize}
\end{itemize}

\section*{Lista de gráficos:}

\begin{itemize}

    \item Título de figura o esquema 1..... Nº de pag.

    \item Título de figura o esquema 2..... Nº de pag.

\end{itemize}

\section*{Introducción}

Introducción al problema, importancia y objetivos.

\section*{Desarrollo:}
\begin{itemize}

    \item Toda información de importancia.

    \item Detallar explicar con vocabulario acorde. 

    \item Citar textos, poner opiniones de personas. Tiene que ser claro y
        preciso, también van a ir las imágenes y/o esquemas.

\end{itemize}

\section*{Conclusión:}

Retomar y analizar lo que se dijo previamente en el desarrollo y demostrar que
se cumplió con el objetivo y/o una conclusión final 

\section*{Bibliografía:}

Material bibliográfico detallado, si es un libro editorial, nombre del libro,
autor,etc. Si es pagina de Internet poner el link y fecha de consulta.

\end{minipage}
}
}

\end{document}
